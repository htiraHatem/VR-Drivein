\documentclass[10pt,a4paper]{article}
\usepackage[german]{babel} 
\usepackage[utf8]{inputenc}
\usepackage{geometry}
\geometry{
	a4paper,
	left=30mm,
	right=30mm,
	top=15mm,
	bottom=30mm
}
\usepackage{amsmath}
\usepackage{amsfonts}
\usepackage{amssymb}
\usepackage{graphicx}
\usepackage{array}
\usepackage[
backend=biber]{biblatex}

\newcommand{\horrule}[1]{\rule{\linewidth}{#1}}



\addbibresource{refs.bib}

\title{	
	\normalfont \normalsize 
	\horrule{0.5pt} \\[0.4cm]
	\huge Computergrafik \& Animation: Teilleistung 1 \\
	\horrule{0.5pt} \\[0.5cm]
}

\author{
	Hatem Htira	\\ \small (1978226)
	\and
	Katharina Lochmüller\\ \small 1775944()
	\and
	Carina Walker\\ \small (1966493)
}

\begin{document}
\maketitle

\section{A1}
s. Abgabe

\section{A2}
s. Abgabe

\section{A3}
\subsection{Erstellung von Transformationsmatritzen}
\subsubsection{Erstellen Sie jeweils eine Transformationsmatrix, um folgende Transformationen vorzunehmen}
\begin{itemize}
	\item  Verschiebung um 6 in X- und -4 in Z-Richtung
	
	$T = \begin{bmatrix}
	1 & 0 & 0 & 6 \\ 
	0 & 1 & 0 & 0 \\ 
	0 & 0 & 1 & -4 \\ 
	0 & 0 & 0 & 1
	\end{bmatrix}  $
	
	\item  Skalierung um den Faktor 3 in Y- und Z-Dimension
	
	$ S = 	\begin{bmatrix}
			1 & 0 & 0 & 0 \\ 0 & 3 & 0 & 0 \\ 0 & 0 & 3 & 0 \\ 0 & 0 & 0 & 1
			\end{bmatrix} $ 
	
	\item  Rotation um 40\degree\ um die Y-Achse
	
	$ R_y = \begin{bmatrix}
				\cos40\degree & 0 & \sin40\degree & 0 \\
				0	& 1 & 0 & 0\\
				-\sin40 & 0 & \cos40 & 0\\
				0 & 0& 0& 1
	\end{bmatrix}$ mit $\cos40\degree \approx 0.766$ und $\sin40\degree \approx 0.643$
	
	\item Verschiebung um 2 in X- und Z-Richtung, anschließend Rotation von 45\degree um Y-Achse
	
	$
		T = \begin{bmatrix}
			1 & 0 & 0 & 2\\
			0 & 1 & 0 & 0\\
			0 & 0 & 1 & 2\\
			0 & 0 & 0 & 1
		\end{bmatrix}$
		
	$	R_y = \begin{bmatrix}
		\cos45\degree & 0 & \sin45\degree & 0\\
		0 & 1 & 0 & 0\\
		-\sin45\degree & 0 & \cos45\degree & 0\\
		0 & 0 & 0 & 1
		\end{bmatrix}
	$
	
	$\Rightarrow (R_y \cdot T) = \begin{bmatrix}
		\dfrac{\sqrt{2}}{2} & 0 & \dfrac{\sqrt{2}}{2} & 2\sqrt{2}\\
		0 & 1 & 0 & 0\\
		0 & 0 & \dfrac{\sqrt{2}}{2} & \sqrt{2}\\
		-\dfrac{\sqrt{2}}{2} & 0 & 0 & 1-\sqrt2 
	\end{bmatrix}$
	
	\item Rotation von 60\degree\ um x-Achse, anschließend Rotation von 125\degree\ um z-Achse
	
	$
		R_x = \begin{bmatrix}
				1 & 0 & 0 & 0\\
				0 & \cos60\degree & -\sin60\degree & 0 \\
				0 & \sin60\degree & \cos60\degree &  0\\
				0 & 0 & 0 & 1
			 \end{bmatrix}
	$
	$
		R_z = \begin{bmatrix}
			\cos125\degree & -\sin125\degree & 0 & 0 \\
			\sin125\degree & \cos125\degree & 0 &0\\
			0 & 0 & 1 & 0\\
			0 & 0 & 0 & 1 
		\end{bmatrix}
	$

	
	$
	\Rightarrow (R_z \cdot R_x) = \begin{bmatrix}
	\cos125\degree & \dfrac{-\sin125\degree}{2} & \dfrac{\sqrt{3} \sin125\degree}{2} & 0\\
	\sin125\degree & \dfrac{\cos125\degree}{2} & \dfrac{-\sqrt{3} \cos125\degree}{2} & 0\\
	0& \dfrac{\sqrt{3}}{2}& \dfrac{1}{2}&0\\
	0& 0& 0& 1
	\end{bmatrix}
	$
\subsubsection{Was
	unterscheidet lineare von strukturverändernden Transformationen? Geben Sie zwei Beispiele für
	strukturverändernde Transformationen an.}


Für zwei Vektorräume  $U,V$ über einem Körper $\mathbb{K}$ erfüllt die lineare Transformation die Eigenschaften der Additivät und Homogenität. Hierdurch ist es möglich Vektoren aus dem Vektorraum $U$ in den Vektorraum $V$ abzubilden, ohne dass sich die Form der Objekte, die von den Vektoren beschrieben werden, ändert. 

Strukturverändernde Transformationen verstoßen gegen die Eigenschaften der Additivität bzw. Homogenität, d.h. beim Abbildungen eines Vektors $u\in U$ in den Vektorraum $V$ ändert sich auch dessen Struktur, z.B. kann ein Objekt, das durch einen abgebildetne Vektor beschrieben wird, seine Form verändern (etwa bei Verjünung oder Verdrehung). 

\subsubsection{ Geben Sie zwei unterschiedliche Beschreibungsformen für Ebenen an und erläutern Sie kurz, wie diese
	ineinander überführt werden können.}

Eine mögliche Beschreibungsform ist die Koordinatenform $ax+by+cz=d$, welche sich durch Umformung z.B. in die Normalenform $\vec{n}\cdot(\vec{p} - \vec{a})$ bringen lässt.

Es sei eine Ebene beschrieben durch den Normalenvektor $\vec{n}=\begin{bmatrix}
a & b & c
\end{bmatrix}^T$, den Aufpunkt $\vec{a} = 
\begin{bmatrix}
x & y & z
\end{bmatrix}^T$ und einen beliebigen Punkt auf der Ebene $\vec{p} = 
\begin{bmatrix}
x_0 & y_0 & z_0
\end{bmatrix}^T$ 

\begin{equation}
\begin{aligned}
	&\vec{n} \cdot (\vec{p} - \vec{a}) = 0 \\
	&\iff \vec{n}\cdot \vec{p}   = \vec{n} \cdot \vec{a}\\
	&\iff ax + by+ cz = ax_0 + by_0 + cz_0\\
	&\iff ax + by + cz = d\\ mit\,\,\ d = \vec{n}\cdot\vec{a}
\end{aligned}
\end{equation}

\end{itemize}

\subsection{Transformation einer Kugel}
\subsubsection{}
\begin{itemize}
	\item Translationsmatrix $T = \begin{bmatrix}
		1 & 0 & 0 & -2\\
		0 & 1 & 0 & 5\\
		0 & 0 & 1 & 0\\
		0 & 0 & 0 & 1
	\end{bmatrix}
	$
	
	\item Skalierungsmatrix $S = \begin{bmatrix}
	
	1 & 0 & 0 & 0\\
	0 & 1 & 0 & 0\\
	0 & 0 & 4 & 0\\
	0 & 0 & 0 & 1\\
	\end{bmatrix}$
	
	\item[] $\Rightarrow$ da Wechsel der Koordinatensysteme: $(S \cdot T)^{-1} = T^{-1} \cdot S^{-1} \\= \begin{bmatrix}
	1 & 0 & 0 & 0\\
	0 &1 &0 &0\\
	0 &0 &\dfrac{1}{4} &0\\
	0 &0 &0 &1
	\end{bmatrix} \cdot \begin{bmatrix}
	1 & 0 & 0 & 2\\
	0 & 1 & 0 & -5\\
	0 & 0 & 1 & 0\\
	0 & 0 & 0 & 1
	\end{bmatrix} = \begin{bmatrix}
	1 & 0 & 0 & 2\\
	0 & 1 & 0 & -5\\
	0 & 0 & \dfrac{1}{4} & 0\\
	0 & 0 & 0 & 1
	\end{bmatrix}$ 
	\item Der Mittelpunkt der Kugel sei beschrieben durch den Vektor $\vec{p_m} = \begin{bmatrix}
	x_m & y_m & z_m & 1
	\end{bmatrix}^T$, dann ist der Mittelpunkt der Kugel im Weltkoordinatensystem:
	$(S \cdot T)^{-1}\cdot \vec{p_m}
		= \begin{bmatrix}
		x+2 \\
		y-5 \\
		\frac{z}{4} \\
		1
		\end{bmatrix}\\ = \begin{bmatrix}
		x+2 & y-5 & \frac{z}{4}
		\end{bmatrix}^T$
		Da der Ursprung = Mittelpunkt, ergibt sich also: $p_m =  \begin{bmatrix}
		2 & -5 & 0
		\end{bmatrix}^T$ als Mittelpunkt der Kugel im Weltkoordinatensystem.
\end{itemize}

\subsubsection{}
\begin{itemize}
	\item Blickrichtung $n=\overrightarrow{CM_w} = 
	\begin{bmatrix}
		2  & -5 & 0 
	\end{bmatrix}^t - \begin{bmatrix}
	10  & -15 & 10 
	\end{bmatrix}^t = \begin{bmatrix}
	-8  & 10 & -10 
	\end{bmatrix}^T$
\end{itemize}
\end{document}
